\documentclass[14pt, a1paper, portrait]{tikzposter} % font sizes: 12pt, 14pt, 17pt, 20pt and 25pt

%\usetheme{Simple}

\usepackage{graphicx}
\usepackage{graphics}

\usepackage{tikz}
\usetikzlibrary{arrows.meta}

\newcommand{\negvspace}[0]{\vspace{-14.8mm}}
\newcommand{\figscale}[0]{0.5}
\newcommand{\spacing}[0]{4mm}
\newcommand{\blockschrink}[0]{7mm}

\newcommand{\textdesc}[1]{\textit{\textbf{#1}}}
\newcommand{\descitem}[1]{\item \textdesc{#1}}
\newcommand{\includeSVG}[1]{
  \includegraphics[scale=1.0]{./figs/#1.pdf}
}
\newcommand{\includeSVGfs}[1]{
  \includegraphics[width=\paperwidth]{./figs/#1.pdf}
}
\newcommand{\includeBitmap}[2]{
  \includegraphics[width=#2]{./figs/#1}
}

\begin{document}
\title{Research Interests}
\author{Aslak Johansen}
\institute{Software Engineering Section, University of Southern Denmark}
\maketitle[width=48cm]

\begin{columns}
  \column{.5}
  
  \block[bodyoffsety=\blockschrink]{\vspace{-6mm}Introduction}{
    Systems work involving streaming data with requirements on latency and availability.
    
    \vspace{\spacing}
    Relevant qualities:
    \begin{itemize}
      \descitem{Concurrency} The ability to allow work to be defined in units that can be executed in parallel (concurrency models).
      \descitem{Latency} The ability to provide robust low latency (e.g, by bounding concurrency).
      \descitem{Availability} The ability to provide service despite the occurrence of unforeseen events across the lifetime of the deployment (gracefull degradation, hot updates, distribution).
    \end{itemize}
    
    \vspace{\spacing}
    Interesting aspects:
    \begin{itemize}
      \item Metadata.
      \item Query languages and interaction models.
      \item In-network processing.
    \end{itemize}
  }
  
  \block[bodyoffsety=\blockschrink]{\vspace{-6mm}Problem}{
    % problem statement
    It is applicable to any domain where:
    How should a generic interface to building latency-intolerant applications on top of streaming data be constructed. In particular, where:
    \begin{itemize}
      \item There is a significant number of data streams (or historical timeseries data).
      \item There is metadata associated with context shared between multiple data streams.
      \item Multiple applications should operate on top of the same data model.
      \item The application interface should be simple.
      \item On-demand virtual datastreams are relevant.
      \item Availability, concurrency and latency matter.
    \end{itemize}
    
    % initial motivation
    \vspace{\spacing}
    The initial motivation came from the building domain as support for an app ecosystem.
  }
  
  \block[bodyoffsety=\blockschrink]{\vspace{-6mm}Domain Metadata Model Example}{
    \hspace{-3mm}\scalebox{0.509}{
      \includeSVG{metadata16}
    }
  }
  
  \column{.5}
  
  \block[bodyoffsety=\blockschrink]{\vspace{-6mm}Interaction Model}{
    Typical model of polling information model:
    
    \negvspace
    \hspace{-10mm}\scalebox{\figscale}{
      \includeSVG{pollpoll13}
    }
    
    Proposed model of subscribing to the result set of a query over the information model:
    
    \negvspace
    \hspace{-10mm}\scalebox{\figscale}{
      \includeSVG{pushpush18}
    }
  }
  
  \block[bodyoffsety=\blockschrink]{\vspace{-6mm}Technology}{
    \textbf{Query Anatomy} The query format should be an extended version of a standardized query language (e.g., OpenCypher).
    \begin{itemize}
      \descitem{Pattern} A pattern definition to query the information model.
      \descitem{Data Subscription} A list of the data streams from each match site to forward.
      \descitem{Unit Preferences} A mapping from modality to unit for in-flight automatic convertion.
      \descitem{Temporal Range} The temporal range of interest. Historical data should be replayed until live data can be forwarded live. Indicators for the transitions.
    \end{itemize}
    
    \vspace{\spacing}
    \textbf{Distributed Model:} Once the service hosting the model becomes distributed the metadata essentially becomes software-defined.
    \begin{itemize}
      \item Support for improved availability.
      \item Support for model federation (equipment comes with own models, query results can be dependent on physical state).
      \item The subscription acts as a session, allowing for demand-driven soft/virtual sensors.
    \end{itemize}
  }
  
%  \block{Other Areas}{
%    \begin{itemize}
%      \item Information Model Lifecycle Tracking
%      \item Caching-Friendly DAG-Based Data Processing
%      \item Data Processing Introspection
%    \end{itemize}
%  }
  
\end{columns}

\block[bodyoffsety=\blockschrink]{\vspace{-6mm}Design}{
  \begin{tikzpicture}[remember picture,overlay]
      % vars
      \newcommand{\appservice}[0]{15cm}
      \newcommand{\servicesrc}[0]{35cm}
      \newcommand{\width}[0]{54cm}
      \newcommand{\border}[0]{5cm}
      \newcommand{\metadatamodelheight}[0]{4cm}
      
      \newcommand{\domainheight}[0]{17.2cm}
      \newcommand{\domaincolor}[0]{green!80!black}
      \newcommand{\domaintextscale}[0]{1.2}
      \newcommand{\unit}[0]{8mm}
      
      
      % styles
      \tikzstyle{arrow} = [
        very thick,
        -{Latex[length=5mm, width=2mm]},
        >=stealth,
        draw=purple,
      ]
      \tikzstyle{domaindoundary} = [
        very thick,
        dashed,
        draw=\domaincolor,
      ]
      \tikzstyle{stream} = [
        very thick,
        rectangle,
        anchor=north west,
        minimum width=\unit,
        minimum height=\unit,
      ]
      \tikzstyle{broker} = [
        very thick,
        circle,
        anchor=center,
        minimum width=4*\unit,
        minimum height=4*\unit,
      ]
      \tikzstyle{source} = [
        very thick,
        diamond,
        anchor=center,
        minimum width=1.0*\unit,
        minimum height=1.0*\unit,
      ]
      
      % hardpoints
      \coordinate (origo) at (0, 0);
      \coordinate (oapp) at ([xshift=0mm] origo);
      \coordinate (oservice) at ([xshift=\appservice] origo);
      \coordinate (osub) at ([xshift=9*\unit, yshift=-(4.5*\unit+\metadatamodelheight)] oservice);
      \coordinate (odata) at ([xshift=8*\unit] osub);
      \coordinate (osrc) at ([xshift=\servicesrc] origo);
      
      \coordinate (obroker) at ([xshift=(\width-\servicesrc)/2] osrc);
      
      % domains
      \node[anchor=south] (applabel) at ([xshift=\appservice/2, yshift=-\domainheight] oapp) {\scalebox{\domaintextscale}{\textsl{\textcolor{\domaincolor}{Application Domain}}}};
      \draw[domaindoundary] (oservice)--([yshift=-\domainheight] oservice);
      \node[anchor=south] (servicelabel) at ([xshift=(\servicesrc-\appservice)/2, yshift=-\domainheight] oservice) {\scalebox{\domaintextscale}{\textsl{\textcolor{\domaincolor}{Service Domain}}}};
      \draw[domaindoundary] (osrc)--([yshift=-\domainheight] osrc);
      \node[anchor=south] (sourcelabel) at ([xshift=(\width-\servicesrc)/2, yshift=-\domainheight] osrc) {\scalebox{\domaintextscale}{\textsl{\textcolor{\domaincolor}{Source Domain}}}};
      
      
      % application
      \node[
        rectangle,
        ultra thick,
        draw=black,
        anchor=north west,
        minimum width=\appservice-\border,
        minimum height=\domainheight-4*5mm,
      ] (metadatamodel) at ([xshift=5mm, yshift=-5mm] oapp) {\scalebox{2}{\rotatebox{90}{Application}}};
      
      % subscription
      \node[
        rectangle,
        very thick,
        dashed,
        draw=purple,
        anchor=north west,
        minimum width=6*\unit,
        minimum height=7*\unit,
      ] (subscription) at ([xshift=0mm, yshift=0mm] osub) {};
      \node[stream, draw=purple] (metadatamodelNodeA) at ([xshift=\unit, yshift=-3*\unit] osub) {};
      \node[stream, draw=purple] (metadatamodelNodeB) at ([xshift=4*\unit, yshift=-\unit] osub) {};
      \node[stream, draw=purple] (metadatamodelNodeC) at ([xshift=4*\unit, yshift=-5*\unit] osub) {};
      \draw[arrow, draw=purple] (metadatamodelNodeA.west)--([xshift=5mm-(10*\unit+\border)] metadatamodelNodeA.west);
      \draw[arrow, draw=purple] ([yshift=\unit/6] metadatamodelNodeB.west)--([yshift=\unit/6, xshift=5mm-(13*\unit+\border)] metadatamodelNodeB.west);
      \draw[arrow, draw=purple] ([yshift=-\unit/6] metadatamodelNodeB.west) to [out=180,in=45] (metadatamodelNodeA);
      \draw[arrow, draw=purple] (metadatamodelNodeC.west) to [out=180,in=-45] (metadatamodelNodeA);
      \node[anchor=north] (sublabel) at ([yshift=-2mm] subscription.south) {\scalebox{\domaintextscale}{\textsl{\textcolor{purple}{Transformations}}}};
      
      % data
      \node[stream, draw=black!20] (streamA) at ([yshift=3*\unit] odata) {};
      \node[stream, draw=black!40] (streamB) at ([yshift=1*\unit] odata) {};
      \node[stream, draw=black] (streamC) at ([yshift=-1*\unit] odata) {};
      \node[stream, draw=black] (streamD) at ([yshift=-3*\unit] odata) {};
      \node[stream, draw=black] (streamE) at ([yshift=-5*\unit] odata) {};
      \node[stream, draw=black!40] (streamF) at ([yshift=-7*\unit] odata) {};
      \node[stream, draw=black!20] (streamG) at ([yshift=-9*\unit] odata) {};
      \draw[arrow, draw=black] (streamB.west)--([xshift=5mm-(17*\unit+\border)] streamB.west);
      \draw[arrow, draw=black] (streamC.west)--(metadatamodelNodeB.east);
      \draw[arrow, draw=black] (streamE.west)--(metadatamodelNodeC.east);
      
      % metadatamodel
      \node[
        rectangle,
        ultra thick,
        draw=black,
        anchor=north west,
        minimum width=\width-\appservice-\border-5mm,
        minimum height=\metadatamodelheight,
      ] (metadatamodel) at ([xshift=\border, yshift=-5mm] oservice) {\scalebox{2}{Information Model}};
      
      % brokers
      \node[broker, draw=black] (brokerA) at ([yshift=-\metadatamodelheight-0.75*(\domainheight-4*5mm-\metadatamodelheight)/3] obroker) {MQTT};
      \node[broker, draw=black] (brokerB) at ([yshift=-\metadatamodelheight-1.75*(\domainheight-4*5mm-\metadatamodelheight)/3, xshift=4cm] obroker) {WebSocket};
      \node[broker, draw=black] (brokerC) at ([yshift=-\metadatamodelheight-2.75*(\domainheight-4*5mm-\metadatamodelheight)/3] obroker) {Kafka};
      \draw[arrow, draw=black] (brokerA) to [out=170,in=0] (streamA);
      \draw[arrow, draw=black] (brokerA) to [out=180,in=0] (streamB);
      \draw[arrow, draw=black] (brokerA) to [out=190,in=0] (streamC);
      \draw[arrow, draw=black] (brokerB) to [out=175,in=0] (streamD);
      \draw[arrow, draw=black] (brokerB) to [out=185,in=0] (streamE);
      \draw[arrow, draw=black] (brokerC) to [out=175,in=0] (streamF);
      \draw[arrow, draw=black] (brokerC) to [out=185,in=0] (streamG);
      
      % src
      \node[source, draw=black] (brokerAsrcA) at ([xshift=40mm, yshift=0mm] brokerA) {};
      \node[source, draw=black] (brokerAsrcB) at ([xshift=28mm, yshift=28mm] brokerA) {};
      \node[source, draw=black] (brokerAsrcC) at ([xshift=0mm, yshift=40mm] brokerA) {};
      \draw[arrow, draw=black] (brokerAsrcA) to (brokerA);
      \draw[arrow, draw=black] (brokerAsrcB) to (brokerA);
      \draw[arrow, draw=black] (brokerAsrcC) to (brokerA);
      \node[source, draw=black] (brokerBsrcA) at ([xshift=20mm, yshift=18mm] brokerB) {};
      \node[source, draw=black] (brokerBsrcB) at ([xshift=28mm, yshift=0mm] brokerB) {};
      \node[source, draw=black] (brokerBsrcC) at ([xshift=20mm, yshift=-18mm] brokerB) {};
      \node[source, draw=black] (brokerBsrcD) at ([xshift=40mm, yshift=15mm] brokerB) {};
      \node[source, draw=black] (brokerBsrcE) at ([xshift=40mm, yshift=-15mm] brokerB) {};
      \draw[arrow, draw=black] (brokerBsrcA) to (brokerB);
      \draw[arrow, draw=black] (brokerBsrcB) to (brokerB);
      \draw[arrow, draw=black] (brokerBsrcC) to (brokerB);
      \draw[arrow, draw=black] (brokerBsrcD) to (brokerB);
      \draw[arrow, draw=black] (brokerBsrcE) to (brokerB);
      \node[source, draw=black] (brokerCsrcA) at ([xshift=32mm, yshift=12mm] brokerC) {};
      \node[source, draw=black] (brokerCsrcB) at ([xshift=32mm, yshift=-12mm] brokerC) {};
      \draw[arrow, draw=black] (brokerCsrcA) to (brokerC);
      \draw[arrow, draw=black] (brokerCsrcB) to (brokerC);
      
      % interaction edges
      \draw[arrow, draw=teal] ([xshift=-2*\border+5mm, yshift=-\metadatamodelheight/3] metadatamodel.north west)--([yshift=-\metadatamodelheight/3] metadatamodel.north west) node[midway,sloped,above] {\scalebox{1.0}{\textsl{\textcolor{teal}{query-based subscription}}}};
      \draw[arrow, draw=teal] ([yshift=-2*\metadatamodelheight/3] metadatamodel.north west)--([xshift=-2*\border+5mm, yshift=-2*\metadatamodelheight/3] metadatamodel.north west) node[midway,sloped,above] {\scalebox{1.0}{\textsl{\textcolor{teal}{stream of result-set diffs}}}};
      \node[
        rectangle,
        very thick,
        dashed,
        draw=teal,
        anchor=north west,
        minimum width=8*\unit,
        minimum height=10.5*\unit,
      ] (realsub) at ([xshift=-\unit, yshift=1.5*\unit] subscription.north west) {};
      \node[anchor=north] (sublabel) at ([yshift=-2mm] realsub.south) {\scalebox{\domaintextscale}{\textsl{\textcolor{teal}{Subscription}}}};
      \draw[arrow, draw=teal] ([yshift=2.5*\unit-1mm] realsub.north)--(realsub.north) node[midway,left] {\scalebox{1.0}{\textsl{\textcolor{teal}{stream of configuration updates}}}};
  \end{tikzpicture}
  \vspace{17.2cm}
}

\end{document}
